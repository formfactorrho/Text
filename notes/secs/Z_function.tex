
	\noindent
	We describe here how to derive a form of the generalized $Z^{\bf\Delta}_{alm}[s,x^2]$ in particular how to obtain Eq.~(\ref{eq:main_Z_acc}) from Eq.~(\ref{eq:clm}).\\
	We will follow Ref.~\cite{Leskovec:2012gb}.
	The generalized Zeta function is defined as
	\begin{eqnarray}
	Z^{\bf\Delta}_{lm}[s,x^2]&=&\sum_{{\bf r}\in P_{\bf \Delta}}\frac{r^l\,Y_{lm}(\hat {\bf r})}{(r^2-x^2)^s}\, ,\label{eq:Zdelta}
	\end{eqnarray}
	where the domain $P_{\bf \Delta}$ is given by
	\begin{eqnarray}
	P_{\Delta}= \left\{{\bf r}|{\bf r}={\bm \gamma}^{-1}({\bf n}-\frac{1}{2}{\bf \Delta}),{\bf n}\in Z^{3}\right\}\, \label{eq:Pd},
	\end{eqnarray}
	with
	\begin{eqnarray}
	{\bf \Delta}&=&\frac{{\bf P}L}{2\pi}\left(1+\frac{m_{a1}^1-m_{a2}^2}{E^{*2}}\right)\, .
	\end{eqnarray}
	Using the following integral definition of the Gamma function (valid only if $Re(s)>0$)
	\begin{eqnarray}
	\Gamma(s)&=&a^s\int_0^\infty dt t^{s-1}e^{-ta}\, ,
	\end{eqnarray}
	for $a=r^2-x^2$ we can rewrite Eq.~(\ref{eq:Zdelta}) as
	\begin{eqnarray}
	Z_{lm}^{\bf \Delta}[s,x^2]&=&\frac{1}{\Gamma(s)}\sum_{{\bf r}\in P_{\bf \Delta}}r^lY_{lm}(\hat{{\bf r}})\int_0^\infty dt t^{s-1}e^{-t(r^2-x^2)}\\
	&=&\frac{1}{\Gamma(s)}\sum_{{\bf r}\in P_{\bf \Delta}}r^lY_{lm}(\hat{{\bf r}})
	\left[\int_0^1 dt t^{s-1}e^{-t(r^2-x^2)}+\int_1^\infty dt t^{s-1}e^{-t(r^2-x^2)}\right]\label{eq:Zexp}.\nonumber\\
	&=&I^{\bf\Delta}_{lm}[s,x^2]+J^{\bf\Delta}_{lm}[s,x^2]\, ,
	\end{eqnarray}
	with
	\begin{eqnarray}
	J^{\bf \Delta}_{lm}[s,x^2]&=&\frac{1}{\Gamma(s)}\sum_{{\bf r}\in P_{\bf \Delta}}r^lY_{lm}(\hat{{\bf r}})\int_1^\infty dt t^{s-1}e^{-t(r^2-x^2)}\, ,
	\end{eqnarray}
	and similarly for $I^{\bf\Delta}_{lm}[s;x^2]$.
	For $s=1$ the function $J^{\bm\Delta}_{lm}[s;x^2]$ is easily evaluated
	\begin{eqnarray}
	J^{\bm\Delta}_{lm}[s,x^2]&=&\sum_{{\bf r}\in P_{\bf \Delta}}r^lY_{lm}(\hat{{\bf r}})\frac{e^{-(r^2-x^2)}}{r^2-x^2}\, .
	\end{eqnarray}
	For reasons that will be clear later it is convenient to retain the general $s$ dependence in $I^{\bf\Delta}_{lm}[s,x^2]$.
	We also define, for convenience, the following function 
	\begin{eqnarray}
	F({\bf r})&=&r^lY_{lm}(\hat{{\bf r}})\, e^{-tr^2}\, ,\label{eq:bigF}
	\end{eqnarray}
	and therefore we can write 
	\begin{eqnarray}
	I^{\bf\Delta}_{lm}[s,x^2]&=& \frac{1}{\Gamma(s)}\int_0^1dt\,t^{s-1}e^{tx^2}\sum_{{\bf r}\in P_\Delta }F({\bf r})\, .
	\end{eqnarray}
	We recall that ${\bf r}\in P_\Delta$ is a function of ${\bf n}\in\mathbb{Z}^3$ (as it can be seen from Eq.~(\ref{eq:Pd}))
	 equivalent to $\sum_{\bf n\in \mathbb{Z}^3}F({\bf r}({\bf n}))$ and therefore
	 \begin{eqnarray}
	 \sum_{{\bf r}\in P_\Delta }F({\bf r})&=&\sum_{\bf n\in \mathbb{Z}^3}F({\bf r}({\bf n}))=\sum_{\bf n\in \mathbb{Z}^3}f({\bf n})\, ,\label{eq:sums}
	 \end{eqnarray}
	 where we have defined the function $f({\bf n})$. 
	 Using Poisson summation formula we can rewrite the sum of Eq.~(\ref{eq:sums}) as
	\begin{eqnarray}
	\sum_{\bf n\in \mathbb{Z}^3}F({\bf r}({\bf n}))=\sum_{\bf n\in \mathbb{Z}^3}f({\bf n})&=&\sum_{\bf n\in \mathbb{Z}^3}\int d{\bf y}f({\bf y})e^{i2\pi{\bf n}\cdot{\bf y}}=\sum_{\bf n\in \mathbb{Z}^3}\int d{\bf y}F({\bf r}({\bf y}))e^{i2\pi{\bf n}\cdot{\bf y}}\, .
	\end{eqnarray}
	where in the last passage is ${\bf r}={\bm \gamma}^{-1}({\bf y}-\frac{1}{2}{\bf \Delta})$.
	Remembering Eq.~(\ref{eq:bigF}) we can rewrite the previous expression as
	\begin{eqnarray}
	\sum_{\bf n\in \mathbb{Z}^3}f({\bf n})&=&\sum_{\bf n\in \mathbb{Z}^3}\int d{\bf y}|{\bf r}({\bf y})|^lY_{lm}(\hat{{\bf r}}({\bf y}))\, e^{-t|{\bf r}({\bf y})|^2+i2\pi{\bf n}\cdot{\bf y}}\, .
	\end{eqnarray}
	It is convenient to 
	perform the change of variables
	\begin{eqnarray}
	{\bf y}&=&{\bm \gamma}{\bf r}+\frac{1}{2}{\bf \Delta}\, ,
	\end{eqnarray}
	that leads to the following change in the integration measure $d{\bf y}=\gamma d{\bf r}$.  Therefore we have
	\begin{eqnarray}
	f({\bf n})&=& e^{i\pi{\bf n}\cdot{\bf\Delta}}\gamma \int d{\bf r}r^lY_{lm}(\hat{\bf r})e^{-tr^2+i2\,\pi{\bf n}\cdot({\bm \gamma}{\bf r})}\nonumber\\
	&=& e^{i\pi{\bf n}\cdot{\bf\Delta}}\gamma \int_0^\infty dr r^{2+l} e^{-tr^2}\int d\hat{\bf r}Y_{lm}(\hat{\bf r})e^{-i{\bf k}\cdot{\bf r}}\, ,
	\end{eqnarray}
	where in the last passage we have defined, for convenience, ${\bf k}=-2\pi{\bm \gamma}^T{\bf n}$. Using the well known relation for $e^{-i{\bf k}\cdot{\bf r}}$
	\begin{eqnarray}
	e^{-i{\bf k}\cdot{\bf r}}&=&4\pi\sum_{l^\prime=0}^{\infty}\sum_{m^\prime=-l^\prime}^{l^\prime}(-i)^{l^\prime}Y_{l^\prime m^\prime}({\hat {\bf k}})Y_{l^\prime m^\prime}(\hat{{\bf r}})^*j_{l^\prime}(kr)\, ,
	\end{eqnarray}
	we obtain
	\begin{eqnarray}
	f({\bf n})&=&4\pi (-i)^le^{i\pi{\bf n}\cdot{\bf\Delta}}\gamma\, Y_{lm}(\hat{{\bf k}})\underbrace{\int_0^\infty dr r^{2+l} e^{-tr^2}j_l(kr)}_{f_l(t)}\, .\label{eq:fn_final}
	\end{eqnarray}
	{\color{red}
	The Hankel transform can be evaluated using MATHEMATICA (as Felipe did) or SYMPY (as I did)
	\begin{eqnarray}
	f_l(t)=\frac{k^l}{2^{l+1}t^{l+3/2}}\Gamma(l+3/2)\frac{_1F_1(l+3/2,l+1,-k^2/(4t))}{\Gamma(l+1)}\, ,\label{eq:us_fl}
	\end{eqnarray}
	where $_1F_1(a,b,c)$ is the hypergeometric function of the first kind and can be written as
	\begin{eqnarray}
J=\Gamma(l+3/2)\frac{_1F_1(l+3/2,l+1,-k^2/(4t))}{\Gamma(l+1)}&=&\frac{\Gamma(l+3/2)}{\Gamma(l+1)}\sum_{i=0}^{\infty}\frac{(l+3/2)_i}{(l+1)_i}\frac{z^i}{i!}\nonumber\\
&=&\sum_{i=0}^{\infty}\frac{\Gamma(l+i+3/2)}{\Gamma(l+i+1)}\frac{z^i}{i!}\, ,
	\end{eqnarray}
where we have defined $z=-k^2/(4t)$, and we have used the definition of the Pochhammer symbols i. e. $(l+3/2)_i=\Gamma(l+3/2+i)/\Gamma(l+3/2)$, and similar for $(l+1)_i$.}\\
	In Ref.~\cite{Leskovec:2012gb} the Hankel transform $f_l(t)$ is stated to be
\begin{eqnarray}
f_l(t)&=&\frac{e^{-k^2/(4\,t)}}{4\,\pi}\left(\frac{k}{2t}\right)^l\left(\frac{\pi}{t}\right)^{3/2}\, .\label{eq:fl_lesko}
\end{eqnarray}
{\color{red}
The expressions in Eq.~(\ref{eq:us_fl}) and Eq.~(\ref{eq:fl_lesko}) are the same if and only if
\begin{eqnarray}
J&=&\frac{\sqrt{\pi}}{2}e^{-z}\, ,
\end{eqnarray}
and since $\Gamma(3/2)/\Gamma(1)=\sqrt{\pi}/2$ that would imply 
\begin{eqnarray}
\sum_{i=0}^{\infty}\frac{\Gamma(l+i+3/2)}{\Gamma(l+i+1)}\frac{z^i}{i!}&=&\Gamma(3/2)e^{-z}\, ,
\end{eqnarray}
and SEEMS to imply the following identity
\begin{eqnarray}
\frac{\Gamma(l+i+3/2)}{\Gamma(l+i+1)}=\frac{\Gamma(3/2)}{\Gamma(1)} \quad \forall l,i\in \mathbb{N}\label{eq:wrong_id}
\end{eqnarray}
this cannot be true. Indeed we have a counterexample if first we consider the case $l+i=0$ the l.h.s of Eq.~(\ref{eq:wrong_id}) is $\sqrt{\pi}/2$ and if then we consider the case $l+i=1$ the l.h.s of Eq~(\ref{eq:wrong_id}) is $3\sqrt{\pi}/8$.\\ Remains to see if the two expressions give significant numerical differences.

}
 So proceeding with Eq.~(\ref{eq:fl_lesko}) we finally obtain
	\begin{eqnarray}
	f({\bf n})&=&(-i)^le^{i\pi{\bf n}\cdot{\bf\Delta}}\gamma\, Y_{lm}(\hat{{\bf k}})e^{-k^2/(4\,t)}\left(\frac{k}{2t}\right)^l\left(\frac{\pi}{t}\right)^{3/2}\, .
	\end{eqnarray}
	Therefore $I^{\bf\Delta}_{lm}[s,x^2]$ can now be written as
	\begin{eqnarray}
		I^{\bf\Delta}_{lm}[s,x^2]&=&\frac{(-i)^l\pi^{3/2}\gamma}{\Gamma(s)}\int_0^1dt\,t^{s-5/2}e^{tx^2}\sum_{{\bf n}\in\mathbb{Z}^3}e^{i\pi{\bf n}\cdot{\bf \Delta}}Y_{lm}(\hat{\bf k})e^{-k^2/(4\,t)}\left(\frac{k}{2t}\right)^l\, .
		%&=&\frac{(-i)^l\pi^{3/2}\gamma}{\Gamma(s)}\int_0^1dt\,t^{s-5/2}e^{tq^2}\nonumber\\
		%&+&\frac{(-i)^l\pi^{3/2}\gamma}{\Gamma(s)}\int_0^1dt\,t^{s-5/2}e^{tq^2}\sum_{{\bf n}\in\mathbb{Z}^3/\{0\}}e^{i\pi{\bf n}\cdot{\bf \Delta}}Y_{lm}(\hat{\bf k})e^{-k^2/t}\left(\frac{k}{2t}\right)^l
	\end{eqnarray}
	We note that the term ${\bf n}={\bf 0}$ in the above sum is different from zero only for $l=m=0$.
	It is important to note now that for the case of interest here ($s=1$) there is a diverrgence for ${\bf n}=0$ and $l=m=0$.
	Indeed we have
	\begin{eqnarray}
	({\mbox term\,\, in\,\,}I^{\bf\Delta}_{00}[s,x^2] {\mbox \,\,with\,\,n=0 })&=&\frac{\pi\gamma}{2\Gamma(s)}\int_0^1dt t^{s-5/2}e^{tq^2}\nonumber\\
	&=&\frac{\pi\gamma}{2\Gamma(s)}\int_0^1dt\left[t^{s-5/2}(e^{tq^2}-1)+t^{s-5/2}\right]\, ,\label{eq:term_div}
	\end{eqnarray} 
	The first integral for $s=1$ becomes
	\begin{eqnarray}
	\int_0^1dt\frac{e^{tq^2}-1}{t^{3/2}}\, ,
	\end{eqnarray}
	which is finite. The second integral in Eq.~\ref{eq:term_div} is divergent for $s=1$ (trivially 
	$\int_0^1dt/t^{3/2}=\infty$). Following Ref.~\cite{Leskovec:2012gb} we consider the function
	\begin{eqnarray}
	g(s)&=&\int_0^1dtt^{s-5/2}=\frac{1}{s-3/2}\, ,\label{eq:g_s}
	\end{eqnarray}
	for $s>3/2$.which is finite. We can have a finite value of $g(s)$ for $s=1$ if we analitically continue  Eq.~(\ref{eq:g_s}) to $s=1$. Using this procedure (and evaluating the r.h.s. of Eq.~(\ref{eq:g_s})) we have $g(s=1)=-2$.\\
	Finally collecting everything we have
	\begin{eqnarray}
	Z^{\bf\Delta}_{lm}(s=1,x^2)&=&\sum_{{\bf r}\in P_\Delta}\frac{e^{-(r^2-x^2)}}{r^2-x^2}r^lY_{lm}(\hat{{\bf r}})+\gamma(-i)^l\pi^{3/2}\int_0^1dt\,\frac{e^{tx^2}}{t^{3/2}}\sum_{{\bf n}\in\mathbb{Z}^3/\{0\}}e^{i\pi{\bf n}\cdot{\bf \Delta}}Y_{lm}(\hat{\bf k})e^{-k^2/(4t)}\left(\frac{k}{2t}\right)^l\nonumber\\
	&+&\frac{\pi\gamma}{2}\delta_{l,0}\delta_{m,0}\left[\int_0^1 dt\,\frac{(e^{tx^2}-1)}{t^{3/2}}-2\right]\, ,
	\end{eqnarray}
	which agrees with Eq.~(\ref{eq:main_Z_acc}) {\color{red}except for the sign in the exponent 
	$e^{i\pi{\bf n}\cdot{\bf \Delta}} $}.
	 
