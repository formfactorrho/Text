\section{Quantization Conditions}
The relativistic scattering amplitude, $\mathcal{M}$, can be written in terms of the $S$-matrix. When there are N-channels open, both the scattering amplitude and the $S$-matrix are $N\times N$ size matrices. It is convenient to introduce a diagonal matrix, $\mathbb{P}=\text{diag}(\sqrt{\xi_1 q^{*}_1},\sqrt{\xi_2 q^{*}_2},\ldots,\sqrt{\xi_N q^{*}_N})/\sqrt{4\pi E^*}$, where $\xi_a$ is 1/2 if the two particles in the $a{th}$ channel are identical and 1 otherwise. The CM energy $E^*$ is defined in terms of total laboratory frame energy and momentum, $E$ and $\mathbf{P}$, as $E^*=\sqrt{E^2-\mathbf{P}^2}$. For the $a{th}$ channel with two mesons each having masses $m_{i,1}$ and $m_{i,2}$, the CM relative momentum is
\begin{align}
\label{momentum}
q^{*2}_a=\frac{1}{4}\left(E^{*2}-2(m_{a,1}^2+m_{a,2}^2)+\frac{
(m_{,1}^2-m_{a,2}^2)^2}{E^{*2}}\right),
\end{align}
which simplifies to $\frac{E^{*2}}{4}-m_{a}^2$ when $m_{a,1}=m_{a,2}=m_{a}$. With this, the scattering amplitude can be written as 
\begin{align}
\label{eq:relscatamp}
i\mathcal{M}&=\mathbb{P}^{-1}~(S-\mathbb{I}_N)~\mathbb{P}^{-1},\\
%\Rightarrow S&=\mathbb{I}_N+i\mathbb{P}\mathcal{M}\mathbb{P}
\end{align}
where $\mathbb{I}_N$ is the N-dimensional identity. 
 
When a single channel is open, composed of a spinless particles, the S-matrix is a diagonal unitary matrix. The diagonal elements correspond to the different angular momenta that the two particles can carry.  Conservation of angular momentum requires the S-matrix to be diagonal in $\ell$. Azimuthal symmetry tells us that all $m$ components are identical. In summary, the S-matrix satisfies the following expression,
\begin{eqnarray}
\mathcal{S}_{\ell,m;\ell',m'}
=\delta_{\ell\ell'}\,\delta_{m m'}\,\mathcal S_{\ell}.
\end{eqnarray}
 which can be parametrized by a single function, the scattering phase shift,
 \begin{align}
 S_{\ell}=e^{2i\delta_\ell}.
\end{align}
From this and Eq.~\ref{eq:relscatamp}, one can arrives at 
 \begin{align}
\mathcal M_{\ell}=\frac{8\pi E^*}{\xi}\frac{1}{q^*\cot\delta_{\ell}-iq^*} \,.
\label{eq:scatamp}
 \end{align}
{\raul [Task \#1: Proof this]}
 
 
The master equation that defines the relationship between the finite volume spectrum and the infinite volume scattering parameters is \cite{Luscher:1986pf, Rummukainen:1995vs, Kim:2005gf, Briceno:2012yi, Hansen:2012tf, Briceno:2014oea}
\begin{align}
\det[F^{-1}(P,L) + \mathcal M(P)]=
{\rm{det}}_{\rm{oc}}\left[\rm{det}_{\rm{pw}}\left[F^{-1}(P,L) + \mathcal M(P)\right]\right]=0 \,.
\label{eq:QC}
\end{align}

where the determinant $\rm{det}_{\rm{oc}}$ is over the N open channels and the determinant $\rm{det}_{\rm{pw}}$ is over the partial waves, and both $\mathcal{M}$ and $\delta \mathcal{G}^V$ functions are evaluated on the on-shell value of the momenta. $F$ is a diagonal matrix in the number of open channels, but it is non-diagonal in angular momentum, 
\begin{equation}
\label{eq:Fscdef}
F_{a \ell m ; a' \ell' m'}(P,L)  \equiv \delta_{aa'}\xi_a 
\left[\frac{1}{L^3}\sum_{\mathbf{k}}\hspace{-.5cm}\int~\right]
\frac{ 4 \pi  Y_{\ell m }(\hat {\textbf k}^*_a)Y^*_{\ell' m'}(\hat {\textbf k}^*_a)  }{2 \omega_{a1} 2 \omega_{a2}(E -  \omega_{a1} - \omega_{a2} + i \epsilon )} \left (\frac{k^{*}_a}{q^*_a} \right)^{\ell+\ell'} \,,
\end{equation}
where we are adopting the compact notation of Ref.~\cite{Briceno:2015tza}
\begin{equation}
 \left[ \frac{1}{L^3}\sum_{\mathbf{k}}\hspace{-.5cm}\int~\right]f(\textbf{k})\equiv\bigg [ \frac{1}{L^3} \sum_{\textbf k \in (2 \pi/L) \mathbb Z^3} - \int \frac{d\textbf k}{(2 \pi)^3} \bigg ] f(\textbf{k})\,.
\end{equation}


In Sec.~\ref{app:Fnum} we show that this can be written as 
\begin{equation}
F_{a \ell m,a' \ell' m'}(P,L)= \delta_{aa'}\frac{iq^*_a}{8\pi E^*}\xi_a \left[\delta_{\ell\ell'}\delta_{m m'} +i\sum_{\ell'' m''}\frac{(4\pi)^{3/2}}{q_a^{*({\ell''}+1)}}c_{a \ell'' m''}^{{\pmb \Delta}}(q_a^{*2};{L})  \int d\Omega~Y^*_{ \ell m}(\hat {\textbf k}^*_a) Y^*_{\ell'' m''}(\hat {\textbf k}^*_a) Y_{\ell' m'}(\hat {\textbf k}^*_a) \right],
\label{eq:Ffunct_KSS}
\end{equation}
where $\Delta \equiv \frac{\textbf P L}{2 \pi} \left (1 + \frac{m_{a1}^2 - m^{a2}_2}{E^{*2}} \right)$ and $c_{a \ell'' m''}^{{\pmb \Delta}}$ are defined in Sec.~\ref{app:Fnum}.

In this work, we will primarily be focused on the case where there is a single channel open composed of two spinless particles that are degenerate. In particular, we will only consider $\pi\pi$ systems, with emphasis on the $\rho$ channel. This means that $\Delta \equiv \frac{\textbf P L}{2 \pi} =\textbf{d}$. Given that it will be surpperflous will drop the channel index $``a"$. Finally, in this limit the determinant appearing in Eq.~\ref{eq:QC} reduces to one over partial waves alone
\begin{align}
\det[F^{-1}(P,L) + \mathcal M(P)]=
\rm{det}_{\rm{pw}}\left[F^{-1}(P,L) + \mathcal M(P)\right]=0 \,.
\label{eq:QC_onech}
\end{align}


%and the function $c^{\textbf{P}}_{lm}$ is defined as~\cite{Davoudi:2011md, Fu:2011xz, Leskovec:2012gb}
% \begin{align}
% 	c^\mathbf{d}_{lm}(q^{*2};L)=\frac{\sqrt{4\pi}}{\gamma L^3}\left(\frac{2\pi}{L}\right)^{l-2}\mathcal{Z}^\mathbf{d}_{lm}[1;(k^*_j {L}/2\pi)^2],
%\hspace{1cm} 
%\mathcal{Z}^\mathbf{d}_{lm}[s;x^2]=\sum_{\mathbf{n}}\frac{\mathbf{r}^lY_{l,m}(\mathbf{r}_n)}{(r^2_n-x^2)^s},
%\end{align}
%%
%where $\mathbf{\mathbf{r}_n}=\frac{1}{\gamma}(\mathbf{n}_{||}-\alpha\mathbf{d})+\mathbf{n}_{\perp}$, $\gamma=E/E^*$, $E=\sqrt{P^2+E^{*2}}$, $\mathbf{n}$ is an integer triplet, $\mathbf d$ is the normalized boost vector $\mathbf d=\mathbf{P}L/2\pi$, and $\alpha=\frac{1}{2}\left[1+\frac{m_1^2-m_2^2}{E^{*2}}\right]$.
%The numerical evaluation $\mathcal{Z}$-function can be accelerated by writting it as~\cite{Leskovec:2012gb}
% \begin{align}
%\mathcal{Z}^\mathbf{d}_{lm}[1;x^2]&=\sum_{\mathbf{n}}\frac{e^{- (r^2_n-x^2)}}{r^2_n-x^2}\mathbf{r}_n^lY_{l,m}(\mathbf{r}_n)
%+\delta_{l,0}~Y_{00}~\gamma~\pi^{3/2}~\left[2x^2\int^{1}_0dt\frac{e^{tx^2}}{\sqrt{t}}-2{e^{ x^2}}\right]\nonumber\\
%&+~\gamma~i^l\sum_{\mathbf{w}\neq0}~e^{i2\pi\alpha\mathbf{w}\cdot \mathbf{d}}~|\hat{\gamma}\mathbf{w}|^l~Y_{l,m}(\Omega_{\hat{\gamma}\mathbf{w}})\int^{1}_0dt\left(\frac{\pi}{t}\right)^{3/2+l}e^{tx^2}e^{-\pi^2|\hat{\gamma}\mathbf{w}|^2/t}
%\end{align}
%where 
%\begin{align}
%\hat{\gamma}\mathbf{w}={\gamma}\mathbf{w}_{||}+\mathbf{w}_{\perp}.
%\end{align}
%
%
%Note, both $\textbf{n}$ and $\textbf{w}$ are integer triplets. For example, $\textbf{n}=(n_x,n_y,n_z)$, where each $n_i$ component is an integer running from negative infinity to positive infinity. $\mathbf{w}_{||}$ and $\mathbf{w}_{\perp}$ are defined such that they are parallel and orthogonal to the boost vector, $\textbf{d}$, respectively. In other words, $\mathbf{w}_{\perp}\cdot\textbf{d}=0$, and $\mathbf{w}_{||}=(\mathbf{w}\cdot\textbf{d})\,\textbf{d}=0$.

  


 %%%%%%%%%%%%%%%%%%%%%%%%%%%%%%%%%%%%%%%%%%
\subsection{Simplification for $\pi\pi$ in the S- and P-wave channels}
%%%%%%%
\begin{center}
\begin{table} 
\begin{tabular}{ c c c c} 
\hspace{.1cm}\textbf{d}\hspace{.1cm}& 
 \hspace{.1cm}$(00n)$\hspace{.1cm}& 
 \hspace{.1cm}$(nn0)$\hspace{.1cm}& 
 \hspace{.1cm}$(nnn)$\hspace{.1cm}
 \\\hline
& $\alpha_{20,\mathbb{A}_1}^\mathbf{d} =\frac{2}{\sqrt{5}}$
& $\alpha_{20,\mathbb{A}_1}^\mathbf{d} =-\frac{1}{\sqrt{5}},\hspace{.5cm}
\alpha_{22,\mathbb{A}_1}^\mathbf{d} =-i\sqrt{\frac{6}{5}}$
& $\alpha_{22,\mathbb{A}_1}^\mathbf{d} =-2i\sqrt{\frac{6}{5}}$
 \\ 
 & $\alpha_{20,\mathbb{E}}^\mathbf{d} =-\frac{1}{\sqrt{5}}$
  & 
 $\alpha_{20,\mathbb{B}_1}^\mathbf{d} =-\frac{1}{\sqrt{5}},\hspace{.5cm}
  \alpha_{22,\mathbb{B}_1}^\mathbf{d} =i\sqrt{\frac{6}{5}}$
 & $\alpha_{22,\mathbb{E}}^\mathbf{d} =i\sqrt{\frac{6}{5}}$\\
 && 
 $\alpha_{20,\mathbb{B}_2}^\mathbf{d} =\frac{2}{\sqrt{5}}$
\vspace{.05cm}\\ \hline
\end{tabular}
\caption{Nonzero values of $\alpha_{20,\Lambda}^\mathbf{d}$ and $\alpha_{22,\Lambda}^\mathbf{d}$ for ${\mathbf{d}^2}\leq3$. For the $\mathbb{T}_1^-$ irrep of ${O}^D_h$, the $c^\mathbf{d}_{2m}$ vanish, therefore there is no need to define $\alpha_{2m,\Lambda}^\mathbf{d}$ for this irrep. 
}
\label{table:alphad}
\end{table}
\end{center} 

Let us consider the low-energy behavior of the quantization condition. In particular, we will consider the case where the  the scattering amplitude is dominated by a single angular momentum. Furthermore, given that the $\pi\pi$ must satisfy Bose statistics, they must have a totally symmetric wavefunction. If we consider the isoscalar ($I=0$) or isotensor ($I=2$) channels, where the isospin component of the wavefunction is symmetric, the orbital angular momenta must be even ($\ell=0,2,4,\ldots$). Equivalently, for the isotriplet channel ($I=1$), the angular momenta must be odd even ($\ell=1,3,5,\ldots$). Some particularly interesting examples, are the $\sigma$ resonance, which couples to the couples to the $(I,\ell)=(0,0)$ channel~\cite{Briceno:2016mjc}, and the $\rho$ resonance which couples to the $(I,\ell)=(1,1)$ channel~\cite{Wilson:2015dqa, Dudek:2012xn}. 

First, let us consider the case where the scattering amplitude is dominated by the S-wave, in other words, 
 \begin{align}
 \hspace{-2cm}
\mathcal M
\approx
\left(
\begin{array}{cccccc}
 \mathcal M_{0} 
 \\
& 0
 \\
&& 0
 \\
&&&0
\\
&&&& \ddots
\\
\end{array}
\right). \hspace{1cm}
 \end{align}
Inserting this into Eq.~\ref{eq:QC}, one can show that the quantization condition is fairly simple
\begin{align}
q^{*}\cot\delta_{S}-
{4\pi}
 c^\mathbf{d}_{00}(q^{*2};L)=0.
 \label{eq:Swave}
\end{align}
{\raul [Task \#2: Proof this]}

The P-waves quantization conditions are only slightly more complicated. First, we need to remember that the $\ell=1$ angular momentum is a three-dimensional irrep of the $\mathcal{O}(3)$ group. This implies that the scattering amplitude has three identical components
 \begin{align}
 \hspace{-2cm}
\mathcal M
\approx
\left(
\begin{array}{cccccc}
  0
 \\
& \mathcal M_{1} 
 \\
&& \mathcal M_{1} 
 \\
&&&\mathcal M_{1} 
\\
&&&& \ddots
\\
\end{array}
\right). \hspace{1cm}
 \end{align}
 Inserting these into Eq.~\ref{eq:QC} and block-diagonalizing the matrix inside the determinant, one finds that for each boost vector, $\mathbf{d}$, there are three quantization conditions describing the spectrum. For the symmetry group we will be considering, these can be compactly written as, 
\begin{align}
q^{*}\cot\delta_{P}-
{4\pi}\left(
 c^\mathbf{d}_{00}(q^{*2};L)
+\frac{\alpha_{20,\Lambda}^\mathbf{d}}{q^{*2}} c^\mathbf{d}_{20}(q^{*2};L)
+\frac{\alpha_{22,\Lambda}^\mathbf{d}
}{q^{*2}} c^\mathbf{d}_{22}(q^{*2};L)
\right)=0
\end{align}
where the values of $\alpha_{2m,\Lambda}^\mathbf{d}$ are shown in Table~\ref{table:alphad}. 


When the system is at rest $\alpha_{2m,\Lambda}^\mathbf{d}=0$, there is a single irrep of the cubic group which couples to P-wave channel. This is the $\mathbb{T}_1^-$ irrep, which is three-dimensional. Given that the three-components of the $\mathbb{T}_1^-$ irrep are necessarily degenerate, only one QC is needed to describe its spectrum, which resembles that of the S-wave,
\begin{align}
\mathbb{T}_1^-:\hspace{.5cm} q^{*}\cot\delta_{P}-
{4\pi}
 c^\mathbf{d}_{00}(q^{*2};L)=0.
\end{align}
When you boost the system along the z-axis, you get two different irreps, the $\mathbb{A}_1$ and $\mathbb{E}$. Note, the $\mathbb{A}_1$ is one-dimensional and $\mathbb{E}$ is two-dimensional. More technically, the $\mathbb{A}_1$ couples only to the helicity-0 components of the P-wave amplitude, while $\mathbb{E}$ couples to the helicity 1 and -1 components. The two irreps have  distinct spectra satisfying,
\begin{align}
\mathbb{A}_1&:\hspace{.5cm}q^{*}\cot\delta_{P}-
{4\pi}
 c^\mathbf{d}_{00}(q^{*2};L)
 -
\frac{2}{\sqrt{5}}\frac{4\pi}{q^{*2}}
 c^\mathbf{d}_{20}(q^{*2};L)=0\\
\mathbb{E}&:\hspace{.5cm}q^{*}\cot\delta_{P}-
{4\pi}
 c^\mathbf{d}_{00}(q^{*2};L)
 +
\frac{1}{\sqrt{5}}\frac{4\pi}{q^{*2}}
 c^\mathbf{d}_{20}(q^{*2};L)=0. 
\end{align}
Note, the two components of the $\mathbb{E}$ are degenerate. 

Similarly, for the other irreps. 
%%%%%%%%%%%%%%%%%%%%%%%%%%%%%%%%%%%%%%%%%%%%%%