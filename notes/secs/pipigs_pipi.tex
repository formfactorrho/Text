
 \begin{equation}
\Big | \langle E_{n_f}, \textbf P_f, L \vert  \mathcal J(0)  \vert E_{n_i}, \textbf P_i, L \rangle \Big |^2_L 
=\frac{1}{L^6} {\rm{Tr}}\left[
 \mathcal R(E_{n_i}, \textbf P_i) 
\Wtildf(P_i,P_f,L)
\mathcal R(E_{n_f}, \textbf P_f)  
\Wtildf(P_f,P_i,L)
\right] \,.
  \label{eq:2to2_notdegen}
\end{equation}

\begin{equation}
\label{eq:WtiltoWdf}
 \Wtildf(P_f,P_i,L)  \equiv \Wdf(P_f,P_i) +  \mathcal  M(P_f) \ [G(L) \cdot \w](P_f,P_i) \ \mathcal M(P_i)  \,.
 \end{equation}

where $\omega$ is the one-body amplitude (closely related to the pion form factor)
\begin{align}
\label{eq:Rdef}
\mathcal R(E_{n}, \textbf P) &\equiv  \lim_{P_4 \rightarrow i E_{n}} \left[ - (i P_4 + E_{n}) \frac{1}{F^{-1}(P,L) + \mathcal M(P)}\right] \,.
%%%%%%%%%%%%%%%%%%%%%%%%%%%%%%%%%%%%%%%%%%%%%
%%%%%%%%%%%%%%%%%%%%%%%%%%%%%%%%%%%%%%%%%%%%%
\\
\label{eq:Gdotw}
 [G(L) \cdot \w]_{a \ell_fm_{f}; b\ell_im_{i}}(P_f,P_i) 
 &\equiv 
\sum_{s,t=1,2}  \xi_a \xi_b  G^{st}_{a \ell_f m_{f}, a' \ell_f' m_{f}'; b' \ell_i' m_{i}', b \ell_i m_{i}}(P_f,P_i,L) 
~\w_{a'sb't; \ell_f' m_{f}';  \ell_i' m_{i}'}(P_f;P_i)  \,.
\end{align} 
and
 \begin{multline}
\label{eq:Gmat}
G^{st}_{a \ell_f m_{f},a' \ell_f' m_{f}'; b' \ell_i' m_{i}', b \ell_i m_{i}}(P_f,P_i,L)  \equiv \\
\hspace{-4cm}
\delta_{aa'} \delta_{bb'}
\left[\frac{1}{L^3}\sum_{\mathbf{k}}\hspace{-.5cm}\int~\right]
~\frac{1}{2\omega_{a\slashed s}}
\frac{ 4 \pi  Y_{\ell_fm_{ f}}(\hat {\textbf k}^*_{af})
Y^*_{\ell_f'm_{f}'}(\hat {\textbf k}^*_{af})
 }{2 \omega_{{asf}}(E_f -  \omega_{a\slashed s} - \omega_{asf} + i \epsilon )} 
  \bigg (\frac{k^{*}_{af}}{q^*_{af}} \bigg)^{\ell_f+\ell_f'}
\frac{ 4 \pi  Y_{\ell_i' m_{i}'}(\hat {\textbf k}^*_{bi})
Y^*_{\ell_i m_{i}}(\hat {\textbf k}^*_{bi})
 }{2 \omega_{bti}(E_i -  \omega_{b\slashed t} - \omega_{bti} + i \epsilon )} 
 \bigg (\frac{k^{*}_{bi}}{q^*_{bi}} \bigg)^{\ell_i+\ell_i'}
  \,.
\end{multline}

Here we have a list of things to do, 

\begin{itemize}
\item Simplify these expressions for the single channel case, degenerate masses, etc
\item I think we might want to write things in terms of $G(L) \cdot \w$ in the first place. This avoid introducing a weird spherical harmonic decomposition of $\w$. 
\end{itemize}


 %%%%%%%%%%%%%%%%%%%%%%%%%%%%%%%%%%%%%%%%%%%%%%
 \subsection{Subduction}
 
%%%%%%%%%%%%%%%%%%%%%%%%%%%%%%%%%%%%%%%%%%
\subsection{Simplifying limits}

\label{sec:simplim}

In this section we consider various simplifying limits of the general result, derived in the last section. We begin by taking the energies considered to be very close to the lowest two-particle threshold.   In this case, the infinite-volume quantities $\w$, $\mathcal M$ and $\Wdf$ are all dominated by their S-wave values. We thus drop all higher partial waves in the matrices $w_{a1b1; \ell' m'; \ell  m }(P_f,P_i)$, $\mathcal M_{ab; \ell' m'; \ell m}(P)$ and $\mathcal W_{\mathrm{df};ab;\ell' m'; \ell m}(P_f,P_i)$. The second consequence of near-threshold energies is that only the lowest two-particle channel is open. In discussing this system it is convenient to introduce the shorthand
\begin{align}
w_{11}(P_f,P_i) & \equiv w_{a1b1; 00;00 }(P_f,P_i) \,, \\
\mathcal M (P) & \equiv \mathcal M_{ab; 00; 00}(P)\,, \\
\mathcal W_{\mathrm{df}}(P_f,P_i) & \equiv \mathcal W_{\mathrm{df};ab;00;00}(P_f,P_i) \,.
\end{align}
We comment here that, for a scalar form factor, symmetry and on-shell constraints guarantee that $w$ only depends on $(P_f-P_i)^2$ and thus not on $\textbf k$. In this case, the truncation of $w$ to the S-wave is exact.  
 Since all matrices have reduced to one dimensional, the trace may be dropped from Eq.~(\ref{eq:2to2_notdegen})
\begin{equation}
\label{eq:mainresSwave}
\Big | \langle E_{n_f}, \textbf P_f, L \vert  \mathcal J(0)  \vert E_{n_i}, \textbf P_i, L \rangle \Big |^2_L 
=\frac{1}{L^6}
 \mathcal R(E_{n_i}, \textbf P_i)
\Wtildf(P_i,P_f,L) \mathcal R(E_{n_f}, \textbf P_f) 
 \Wtildf(P_f,P_i,L) \,.
\end{equation}
 In addition, the residue matrix $\mathcal R$ simplifies significantly
\begin{align}
\mathcal R(E_{n}, \textbf P) & =   \left [ \frac{\partial}{\partial E} \left( F^{-1}(P,L) + \mathcal M(P) \right ) \right ]_{E=E_n}^{-1} = -  \left [ \mathcal M^2(P) \frac{\partial}{\partial E} \left( F(P,L) + \mathcal M^{-1}(P) \right ) \right ]_{E=E_n}^{-1} \,, \\
& = - \xi \frac{q^*}{8 \pi E^*}   \left [ \sin^2\! \delta\ e^{ 2i \delta}\  \frac{\partial}{\partial E} \left( \cot\phi^\textbf{d}_{}+ \cot\delta \right ) \right ]_{E=E_n}^{-1}    \,, \\
& = \xi \frac{q^*}{8 \pi E^*} e^{-2i \delta}  \left [  \frac{\partial}{\partial E} \left( \phi^\textbf{d}_{} + \delta \right ) \right ]_{E=E_n}^{-1}    \,,
 \end{align}
where $F = F_{a00;b00}$ is understood and where we have introduced the S-wave L\"uscher pseudophase
 \begin{equation}
 \cot \phi^{\textbf d} = \xi \frac{q^*}{8 \pi E^*} \mathrm{Re}F(P,L) \,.
 \end{equation}
Here we have also used the relation between scattering amplitude $\mathcal M$ and scattering phase shift $\delta$, given in Eq.~(\ref{eq:scatamp}) above. Substituting this result for $\mathcal R$ into Eq.~(\ref{eq:mainresSwave}) and rearranging gives
 \begin{multline}
 \label{eq:swaveWtoFV}
\Big [ e^{-i \delta_i}
\Wtildf(P_i,P_f,L) e^{- i \delta_f} \Big ] \Big [ e^{- i \delta_f}  \Wtildf(P_f,P_i,L) e^{- i \delta_i} \Big ]
\\[5pt] = \frac{8 \pi E^*_f}{q^*_f \xi} \frac{8 \pi E^*_i}{q^*_i \xi}  \left [  \frac{\partial}{\partial E_f} \left( \phi^\textbf{d} + \delta \right ) \right ]_{E_f=E_{f,n}} \left [  \frac{\partial}{\partial E_i} \left( \phi^\textbf{d} + \delta \right ) \right ]_{E_i=E_{i,n}} L^6 \Big | \langle E_{n_f}, \textbf P_f, L \vert  \mathcal J(0)  \vert E_{n_i}, \textbf P_i, L \rangle \Big |^2_L 
 \,.
\end{multline}
We thus see that a naive Lellouch-L\"uscher-like proportionality factor arises between the finite- and infinite-volume quantities. Since the right-hand side of this expression is manifestly pure real, this result also suggest a Watson-like theorem for $\Wtildf$, namely that its complex phases are the strong scattering phases associated with the incoming and outgoing two-particle states. 
 %%%%%%%%%%%%%%%%%%%%%%%%%%%%%%%%%%%%%%%%%%