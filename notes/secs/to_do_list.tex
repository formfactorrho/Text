%\begin{table}
\begin{center}
  \begin{tabular}{ c | c | c | c | c | c | c  }
    \hline
                           & \ \ Alessandro \ \  & \ \  Ra\'ul \ \  & \ \ Bipasha \ \  & \ \  Max \ \  & \ \  Felipe \ \  & \ \  Dave \ \   \\ \hline \\[-10pt] \hline
%
{\bf Toy analyses} &                          &                      &                        &                   &                    &                      \\ \hline
{\em scalar}        &                            &                      &                        &                   &                     &                        \\ \hline
{\em vector}        &                            &                      &                        &                   &                     &                       \\ \hline \\[-10pt] \hline
%
 {\bf General LQCD} &                      &                      &                        &                   &                     &                      \\ \hline
 {\em gen props} &                            & date              & date                & date           &                     &                      \\ \hline
 {\em contractions} &                         & date             & date                & date            &                     &                      \\ \hline \\[-10pt] \hline
%
 {\bf $\pi$ form factor}&                     &                      &                        &                   &                     &                      \\ \hline
 {\em subduction}  &                          & date              & date                &                   &                     &                      \\ \hline
 {\em subtask 2}  &                            &                      &                        &                   &                     &                      \\ \hline \\[-10pt] \hline
%
{\bf $\rho$ form factor} &                  &                      &                        &                   &                    &                      \\ \hline
{\em subduction}      &                      & date              & date                &                   &                     &                        \\ \hline
{\em subtask 2}        &                      &                      &                        &                   &                     &                       \\ \hline \\[-10pt] \hline  
%
{\bf LL code}       &                            &                      &                        &                   &                     &                      \\ \hline
{\em subtask 1}  &                            &                      &                        &                   &                     & date               \\ \hline \\[-10pt] \hline
%
{\bf Evaluate $G$}  &                        &                      &                        &                   &                     &                      \\ \hline
{\em accelerated form} & date          &                      &                        & date           & date            &                      \\ \hline  
%
\end{tabular}
\end{center}

\subsection{Toy analyses}

Perform a toy-model analysis of  the elastic form factor for a scalar system, i.e.~the $\sigma$. Here there is only a single form factor. Then repeat the analysis for a vector system, i.e.~the $\rho$. Here there are three form factors. In the case of the vector things are more complicated, not just because of the finite-volume effects, but also the Lorentz decomposition. Specifically we can no long disentangle the three form-factors independently because different irreps have different energies and virtualities.

\subsection{General LQCD}

\subsection{$\pi$ form factor}

\subsection{$\rho$ form factor}

\subsection{LL code}

\subsection{Evaluate $G$}
%\end{table}

 %%%%%%%%%%%%%%%%%%%%%%%%%%%%%%%%%%%%%%%%%%
%\section{Introduction}
%{\raul we need to do a fake data analysis. It would be good to have some models for the $\pi\pi\gamma^\star\to\pi\pi$ amplitudes. We should do the analysis when there is a single form factor, namely for a scalar channel (like the $\sigma$), and when there are three form factor, namely for the vector channel. }

\clearpage